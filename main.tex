\documentclass[a4paper,10pt]{article}
\usepackage[T1]{fontenc}        % Codifica dei font
\usepackage[utf8]{inputenc}    % Lettere accentate da tastiera
\usepackage[english,italian]{babel}     % Lingua del documento
\usepackage{color, colortbl}
\usepackage{booktabs}
\usepackage{array}
%\usepackage[compact]{titlesec}
\usepackage[margin=0.8in, includefoot]{geometry}
\usepackage{longtable}
\usepackage{graphicx}
\usepackage{amssymb}


\newcolumntype{P}[1]{>{\centering\arraybackslash}p{#1}}
\newcolumntype{S}[1]{>{\arraybackslash}p{#1}}
%\titleformat{\chapter}[display]{\bfseries}{}{-30pt}{\Huge}

\renewcommand{\familydefault}{\sfdefault}
\usepackage[scaled]{helvet}
\begin{document}
\title{%
    Progetto per il corso di Basi di Dati \\
    \large LT in Informatica, Universita degli studi di Padova}
\author{Niccolò Mantovani, 1234567
    \and
    Filippo Pinton, 1234567}
\date{}

\maketitle

\tableofcontents

\section{Abstract}
Abstract che di default occupa una pagina intera, cosi come la tabella dei contenuti


\section{Analisi dei requisiti}
%\section{introduzione}
 %   Introduzione all'analisi dei requisiti

%\section{considerazioni}
 %   \input{src/analisiRequisiti/considerazioni}
 
 Si vuole realizzare una base di dati per rappresentare al meglio il lavoro della cantina \emph{Nome cantina}. Come prodotto principale c'è la bottiglia di vino che è caratterizzata dal tipo di vino contenuto, dall'annata, dal prezzo per bottiglia, dal numero di bottiglie prodotte e vendute ed infine dalla classificazione: vini a denominazione d'origine controllata e garantita (D.O.C.G.); vini a denominazione d'origine controllata (D.O.C.); vini ad indicazione geografica tipica (I.G.T.). Ciascuna tipologia di bottiglia di vino è immagazzinata in magazzini aventi il numero di bottiglie contenute divisi per la colorazione del vino (rosso, bianco, rosato e spumante).
Altro dato importante è utile identificare il vino prodotto, che viene rappresentato tramite il nome, la gradazione alcolica, il tempo di fermentazione, la tipologia di uva con cui è stata prodotta, caratterizzata dalla vedemmia (data raccolta) e dal colore. Importante inoltre tenere traccia se un determinato vino è ancora in fase di produzione o è diventato fuori commercio. 
    
  \section{Strutturazione dei requisiti}
   \begin{center}
	\begin{tabular}{P{16cm}}
		\toprule
		\rowcolor[rgb]{.929, .929, .929} \textbf {\large {FRASI RELATIVE A BOTTIGLIA DI VINO}}                                                                                                                                                                                                                                                                                                                                                                                                                                                                                                     \\
		\midrule
		Una bottiglia di vino è il prodotto finale che viene venduto dalla cantina, ed è la composizione di una bottiglia di vetro, di un tappo e di un tipo di vino. Ogni bottiglia è identificata dal nome del vino e dall'anno (\emph{annata}) di imbottigliamento. Per ogni prodotto è necessario tenere traccia del prezzo, del numero di unità prodotte e vendute, oltre che del tipo di tappo e del tipo di bottiglia. Nel magazzino viene tenuto traccia della quantità delle bottiglie. Le bottiglie di vino possono essere spedite in una determinata quantità negli ordine di vendita \\
		\bottomrule
	\end{tabular}

	\vspace{0.5cm}

	\begin{tabular}{P{16cm}}
		\toprule
		\rowcolor[rgb]{.929, .929, .929} \textbf {\large {FRASI RELATIVE A TIPO DI VINO}}                                                                                                                                                                                                                                                                                                                                                                                          \\
		\midrule
		Relativamente ai tipi di vino, identificati attraverso il nome, rappresentiamo la gradazione alcolica, il tempo di fermentazione e l'uva utilizzata oltre che lo stato produttivo, ovvero se il vino viene prodotto ancora o è, ad oggi, fuori commercio. Importante registrare la classificazione del vino: vini a denominazione d'origine controllata e garantita (D.O.C.G.); vini a denominazione d'origine controllata (D.O.C.); vini ad indicazione geografica tipica (I.G.T.). \\
		\bottomrule
	\end{tabular}

	\vspace{0.5cm}

	\begin{tabular}{P{16cm}}
		\toprule
		\rowcolor[rgb]{.929, .929, .929} \textbf {\large {FRASI RELATIVE A MATERIA PRIMA}}                                                                                                                                                                                                                                                                                                                                                                                                                                                                                                                \\
		\midrule
		Le materie prime che vengono acquistate vengono identificate attraverso un ID univoco. Se la materia prima è l'uva, questa viene rappresentata dall'anno di raccolta, dal nome e dal colore. Se, invece, la materia prima sono i tappi, viene rappresentata dalla forma e dal materiale, infine se la materia prima sono le bottiglie, viene rappresentata dalla capacità e dal colore. Inoltre è presente il dato che identifica la quantità posseduta dalla cantina per i tappi e per le bottiglie, ma non per l'uva poichè tutta l'uva viene utilizzata immediatamente dopo la ricezione. \\
		\bottomrule
	\end{tabular}

	\vspace{0.5cm}

	\begin{tabular}{P{16cm}}
		\toprule
		\rowcolor[rgb]{.929, .929, .929} \textbf {\large {FRASI RELATIVE A VIGNETO}}     \\
		\midrule
		I vigneti sono rappresentati da un ID univoco e dalla sua posizione geografica. \\
		\bottomrule
	\end{tabular}

	\vspace{0.5cm}

	\begin{tabular}{P{16cm}}
		\toprule
		\rowcolor[rgb]{.929, .929, .929} \textbf {\large {FRASI RELATIVE AD AZIENDA}}                                                                                                                                                                                                                                                                                                                                                                                                                                                        \\
		\midrule
		Le aziende si dividono in fornitori, aziende di manutenzione macchinari e negozi interni della cantina. Le aziende sono identificate univocamente attraverso la loro Partita IVA e sono caratterizzate dal nome commerciale, dal cognome e nome del referente oltre che dal numero di telefono, dall'email aziendale e dalla sua posizione geografica (che corrisponde con l'indirizzo di spedizione). Per le aziende che sono fornitori rappresentiamo poi la tipologia di materia prima fornita, sia essa uva, tappi o bottiglie. \\
		\bottomrule
	\end{tabular}

	\vspace{0.5cm}

	\begin{tabular}{P{16cm}}
		\toprule
		\rowcolor[rgb]{.929, .929, .929} \textbf {\large {FRASI RELATIVE AD ORDINE}}                                                                                                                                                                                                                                                                                          \\
		\midrule
		Relativamente ad un ordine rappresentiamo un ID univoco, la data in cui è stato effettuato, il corriere a cui è stata affidata la consegna, la data di spedizione, l'eventuale data di consegna e il prezzo totale (che deve corrispondere al prodotto tra la quantità delle bottiglie di vino acquistate ed il relativo prezzo, sommato al costo della spedizione). \\
		\bottomrule
	\end{tabular}

	\vspace{0.5cm}

	\begin{tabular}{P{16cm}}
		\toprule
		\rowcolor[rgb]{.929, .929, .929} \textbf {\large {FRASI RELATIVE A CORRIERE}} \\
		\midrule
		Il corriere è rappresentato tramite un ID ed un nome commerciale.             \\
		\bottomrule
	\end{tabular}

	\vspace{0.5cm}

	\begin{tabular}{P{16cm}}
		\toprule
		\rowcolor[rgb]{.929, .929, .929} \textbf {\large {FRASI RELATIVE A PRIVATO}}                                                                                                                  \\
		Per i privati, identificati tramite un ID, rappresentiamo il nome, il cognome, il numero di telefono e l'email oltre che l'indirizzo di residenza (che corrisponde con l'indirizzo di spedizione). \\
		\bottomrule
	\end{tabular}

	\vspace{0.5cm}

	\begin{tabular}{P{16cm}}
		\toprule
		\rowcolor[rgb]{.929, .929, .929} \textbf {\large {FRASI RELATIVE AD EVENTO}}                                                                                                                                    \\
		Gli eventi hanno una tematica che è rappresentata da un nostro vino e vengono definiti tramite il titolo dell'evento e il numero di edizione. Ovviamente ogni edizione di un evento ha una data prefissata in cui si svolgerà o si è svolto. \\
		\bottomrule
	\end{tabular}

	\vspace{0.5cm}

	\begin{tabular}{P{16cm}}
		\toprule
		\rowcolor[rgb]{.929, .929, .929} \textbf {\large {FRASI RELATIVE A LINEA PRODUTTIVA}}                                                                                                                                                                                                                                                                                                                                                                                                                                                                                                         \\
		Le linee produttive sono l'ingresso delle materie prime, la pigiatura, la fermentazione, la vinificazione e la svinatura. Oltre a queste esistono altri reparti produttivi come l'imbottigliamento e il magazzino dove le bottiglie di vino vengono divise per la colorazione del proprio vino (rosso, bianco, rosato e spumante). Nel magazzino viene tenuta traccia della quantità delle bottiglie conservate. Tutti questi reparti sono identificati tramite un identificativo univoco. Per ogni linea produttiva sono presenti dei macchinari che aiutano i dipendenti nelle mansioni quotidiane. \\
		\bottomrule
	\end{tabular}

	\vspace{0.5cm}

	\begin{tabular}{P{16cm}}
		\toprule
		\rowcolor[rgb]{.929, .929, .929} \textbf {\large {FRASI RELATIVE A DIPENDENTE}}                                                                                                                                                                                                                                                                                                   \\
		Per ogni dipendete (identificato tramite il codice fiscale) viene registrato il nome e il cognome. Ogni dipendente (esclusi alcuni supervisori) ha un supervisore a cui fare riferimento e ogni linea produttiva è diretta da un supervisore. Per ogni dipendente è necessario tenere traccia dei turni di lavoro effettuati e della linea produttiva in cui i vari turni sono svolti. \\
		\bottomrule
	\end{tabular}

	\vspace{0.5cm}

	\begin{tabular}{P{16cm}}
		\toprule
		\rowcolor[rgb]{.929, .929, .929} \textbf {\large {FRASI RELATIVE A MACCHINARIO}}                                                                                                                                                                                    \\
		I macchinari sono identificati tramite un ID univoco. Ad ogni macchinario viene fatta una manutenzione periodica, è quindi utile tenere traccia della data della prossima manutenzione. I macchinari avranno anche un nome rappresentatativo e la data di acquisto.
		Le manutenzioni hanno un costo e vengono eseguite in una determinata data. Le manutenzioni vengono eseguite da aziende esterne.                                                                                                                                     \\
		\bottomrule
	\end{tabular}

\end{center}
  
  
  

\section{Progettazione concettuale}
\subsection{Analisi entita'}\label{analisi_entita}
    Il grassetto negli attributi indica che quell'attributo è o fa parte di una chiave. \\
Il simbolo $\gets$ identifica una generalizzazione completa, cioè l'\textbf{entita'} a sinistra è una generalizzazione completa delle identità che stanno a destra.
Il simbolo $\Leftarrow$ identifica una generalizzazione parziale, cioè l'\textbf{entita'} a sinistra è una generalizzazione parziale delle identità che stanno a destra.

\begin{verse}
	L'\textbf{entita'} \emph{ProduzioneVino} è una generalizzazione completa delle entita' \emph{IngrMateriePrime, Pigiatura, Fermentazione, Vinificazione, Svinatura} che sono tutte prive di attributi.
	L'\textbf{entita'} \emph{Fornitore}  è una generalizzazione completa delle entita' \emph{FornitoreTappi, FornitureUva, FornitoreBottiglie} che sono tutte prive di attributi
\end{verse}
\begin{verse}
	Le \textbf{entita'} \emph{Fornitore, ProduzioneVino, Imbottigliamento, NegozioInterno, MagBianco, MagRosso, MagRosato, MagSpumante, AzManutenzione} sono prive di attributi.
\end{verse}

\vspace{0.5cm}
\begin{center}
	\begin{tabular}{P{4cm}P{2cm}P{8cm}}
		\multicolumn{3}{c}{\textbf {\large {Vigneto}}} \\
		\toprule
		\rowcolor[rgb]{.929, .929, .929} Attributo & Tipo & Descrizione \\
		\midrule
		\textbf{ID} & INTEGER &  Identifica univocamente il vigneto\\
		\midrule
		Indirzzo & VARCHAR &  Identifica la posizione geografica del vigneto.  Attributo composto: stato, città, provincia, cap, via, numero civico\\
		\bottomrule
	\end{tabular}
	
	\vspace{0.5cm}
	
	\begin{tabular}{P{4cm}P{2cm}P{8cm}}
		\multicolumn{3}{c}{\textbf {\large {Vigna}}} \\
		\toprule
		\rowcolor[rgb]{.929, .929, .929} Attributo & Tipo & Descrizione \\
		\midrule
		\textbf{ID} & INTEGER &  Identifica univocamente la vigna\\
		\midrule
		DataPiantagione & DATE & Data piantagione della vigna \\
		\bottomrule
	\end{tabular}

		\vspace{0.5cm}
	
	\begin{tabular}{P{4cm}P{2cm}P{8cm}}
		\multicolumn{3}{c}{\textbf {\large {TipoUva}}} \\
		\toprule
		\rowcolor[rgb]{.929, .929, .929} Attributo & Tipo & Descrizione \\
		\midrule
		\textbf{Colore} & VARCHAR & Rappresenta il colore dell'uva \\
		\midrule
		\textbf{Nome} & VARCHAR & Rappresenta il nome dell'uva \\
		\bottomrule
	\end{tabular}
	
	\vspace{0.5cm}
	\begin{tabular}{P{4cm}P{2cm}P{8cm}}
		\multicolumn{3}{c}{\textbf {\large {Uva}}} \\
		\toprule
		\rowcolor[rgb]{.929, .929, .929} Attributo & Tipo & Descrizione \\
		\midrule
		Annata & INTEGER & Anno in cui viene raccolta l'uva \\
		\bottomrule
	\end{tabular}

		\vspace{0.5cm}
	
	
\begin{tabular}{P{4cm}P{2cm}P{8cm}}
	\multicolumn{3}{c}{\textbf {\large {Tappo}}} \\
	\toprule
	\rowcolor[rgb]{.929, .929, .929} Attributo & Tipo & Descrizione \\
	\midrule
	Forma & VARCHAR &  Rappresenta il tipo di forma del tappo\\
	\midrule
	Materiale & VARCHAR &  Rappresenta il tipo di materiale del tappo\\
	\midrule
	Quantità & INTEGER &  Rappresenta il numero di tappi per forma e colore posseduti dalla cantina\\
	\bottomrule
\end{tabular}

	\vspace{0.5cm}

\begin{tabular}{P{4cm}P{2cm}P{8cm}}
	\multicolumn{3}{c}{\textbf {\large {Bottiglia}}} \\
	\toprule
	\rowcolor[rgb]{.929, .929, .929} Attributo & Tipo & Descrizione \\
	\midrule
	Colore & VARCHAR &  Rappresenta il colore della bottiglia\\
	\midrule
	Capacità & VARCHAR &  Rappresenta la capacità della bottiglia\\
	\midrule
	Quantità & INTEGER &  Rappresenta il numero di bottiglie per capacità e colore possedute dalla cantina\\
	\bottomrule
\end{tabular}
	\vspace{0.5cm}
	
	\begin{tabular}{P{4cm}P{2cm}P{8cm}}
	\multicolumn{3}{c}{\textbf {\large {MateriaPrima} $\gets$ (\emph{Uva, Tappo, Bottiglia})}} \\
	\toprule
	\rowcolor[rgb]{.929, .929, .929} Attributo & Tipo & Descrizione \\
	\midrule
	\textbf{ID} & INTEGER &  Rappresenta univocamente la materia prima\\
	\bottomrule
\end{tabular}

	\vspace{0.5cm}

\begin{tabular}{P{4cm}P{2cm}P{8cm}}
	\multicolumn{3}{c}{\textbf {\large {Vino}}} \\
	\toprule
	\rowcolor[rgb]{.929, .929, .929} Attributo & Tipo & Descrizione \\
	\midrule
	\textbf{Nome} & VARCHAR & Identifica il nome del vino\\
	\midrule
	GradazioneAlcolica & TINYINT & Identifica il grado di alcol del vino\\
	\midrule
	TempoFermentazione & TINYINT & Rappresenta i giorni che sono serviti per la fermentazione del vino\\
	\midrule
	StatoProduzione & BOOLEAN & Rappresenta se il vino è ancora in produzione\\
	\bottomrule
\end{tabular}


	\vspace{0.5cm}

\begin{tabular}{P{4cm}P{2cm}P{8cm}}
	\multicolumn{3}{c}{\textbf {\large {BottigliaDiVino}}} \\
	\toprule
	\rowcolor[rgb]{.929, .929, .929} Attributo & Tipo & Descrizione \\
	\midrule
	\textbf{Nome} & VARCHAR &  Rappresenta il nome del vino a cui la bottiglia fa riferimento\\
	\midrule
	\textbf{Annata} & INTEGER &  Rappresenta l'anno della vendemmia dell'uva con cui è stato prodotto il vino\\
	\midrule
	Prezzo & DECIMAL &  Identifica il prezzo della bottiglia di vino\\
	\midrule
	Classificazione & ENUM & Tutela i consumatori su alcune caratteristiche del vino\\
	\midrule
	NumBottiglieVendute & INTEGER & Numero di bottiglie per nome e annata di vino vendute\\
	\midrule
	NumBottiglieProdotte & INTEGER &  Numero di bottiglie per nome e annata di vino prodotte\\
	\bottomrule
\end{tabular}


	\vspace{0.5cm}

\begin{tabular}{P{4cm}P{2cm}P{8cm}}
	\multicolumn{3}{c}{\textbf {\large {Ordine}}} \\
	\toprule
	\rowcolor[rgb]{.929, .929, .929} Attributo & Tipo & Descrizione \\
	\midrule
	\textbf{ID} & INTEGER &  Identifica univocamente un ordine di vendita ricevuto\\
	\midrule
	PrezzoTotale & INTEGER &  Rappresenta il prezzo complessivo dell'ordine, formato da prezzo di spedizione più il costo delle bottigle acquistate\\
	\midrule
	Data & DATE &  Data in cui è stato effettuato l'ordine\\
	\bottomrule
\end{tabular}

	\vspace{0.5cm}

\begin{tabular}{P{4cm}P{2cm}P{8cm}}
	\multicolumn{3}{c}{\textbf {\large {Corriere}}} \\
	\toprule
	\rowcolor[rgb]{.929, .929, .929} Attributo & Tipo & Descrizione \\
	\midrule
	\textbf{Id} & INTEGER &  Identifica univocamente il corriere\\
	\midrule
	Nome & VARCHAR &  Rappresenta il nome commerciale del corriere\\
	\bottomrule
\end{tabular}

	\vspace{0.5cm}


\begin{tabular}{P{4cm}P{2cm}P{8cm}}
	\multicolumn{3}{c}{\textbf {\large {Azienda} $\gets$ (\emph{NegozioInterno, Fornitore, AzManutenzione})}} \\
	\toprule
	\rowcolor[rgb]{.929, .929, .929} Attributo & Tipo & Descrizione \\
	\midrule
	\textbf{Id} & INTEGER &  Identifica univocamente l'azienda\\
	\midrule
	NomeReferente & VARCHAR &  Rappresenta il nome del referente dell'azienda\\
	\midrule
	CognomeReferente & VARCHAR &  Rappresenta il cognome del referente dell'azienda\\	\midrule
	PartitaIVA & VARCHAR &  Identifica la partita IVA dell'azienda\\
	\midrule
	Nome & VARCHAR &  Rappresenta il nome dell'azienda\\
	\midrule
	Telefono & VARCHAR &  Rappresenta il numero telefonico dell'azienda\\
	\midrule
	Indirzzo & VARCHAR &  Identifica la posizione geografica cui risiede l'azienda.  Attributo composto: stato, città, provincia, cap, via, numero civico\\
	\midrule
	Email & VARCHAR & Identifica l'email aziendale\\
	\bottomrule
\end{tabular}

\vspace{0.5cm}

\begin{tabular}{P{4cm}P{2cm}P{8cm}}
	\multicolumn{3}{c}{\textbf {\large {Privato}}} \\
	\toprule
	\rowcolor[rgb]{.929, .929, .929} Attributo & Tipo & Descrizione \\
	\midrule
	Nome & VARCHAR &  Rappresenta il nome dell'acquirente privato\\
	\midrule
	Telefono & VARCHAR &  Rappresenta il numero telefonico dell'acquirente privato\\
	\midrule
	Indirzzo & VARCHAR &  Identifica la posizione geografica cui risiede l'acquirente privato.  Attributo composto: stato, città, provincia, cap, via, numero civico\\
	\midrule
	Email & VARCHAR & Identifica l'email dell'acquirente privato\\
	\midrule
	Cognome & VARCHAR &  Rappresenta il cognome dell'acquirente privato\\
	\bottomrule
\end{tabular}

\vspace{0.5cm}

\begin{tabular}{P{4cm}P{2cm}P{8cm}}
	\multicolumn{3}{c}{\textbf {\large {Acquirente} $\gets$ (\emph{Privato}), \large{Acquirente} $\Leftarrow$ (\emph{Azienda})}} \\
	\toprule
	\rowcolor[rgb]{.929, .929, .929} Attributo & Tipo & Descrizione \\
	\midrule
	\textbf{Id} & INTEGER &  Identifica univocamente l'acquirente\\
	\bottomrule
\end{tabular}

\vspace{0.5cm}

\begin{tabular}{P{4cm}P{2cm}P{8cm}}
	\multicolumn{3}{c}{\textbf {\large {Evento}}} \\
	\toprule
	\rowcolor[rgb]{.929, .929, .929} Attributo & Tipo & Descrizione \\
	\midrule
	\textbf{Titolo} & VARCHAR &  Rappresenta il titolo dell'evento\\
	\midrule
	\textbf{Edizione} & INTEGER &  Rappresenta l'edizione dell'evento\\
	\bottomrule
\end{tabular}

\vspace{0.5cm}

\begin{tabular}{P{4cm}P{2cm}P{8cm}}
	\multicolumn{3}{c}{\textbf {\large {Partecipante}}} \\
	\toprule
	\rowcolor[rgb]{.929, .929, .929} Attributo & Tipo & Descrizione \\
	\midrule
	\textbf{Id} & INTEGER &  Rappresenta univocamente il partecipante di un evento\\
	\midrule
	Nome & VARCHAR &  Rappresenta il nome del partecipante\\
	\midrule
	Cognome & VARCHAR &  Rappresenta il cognome del partecipante\\
	\midrule
	Età & TINYINT &  Rappresenta l'età del partecipante, che devo essere maggiore o uguale a 18 anni\\
	\bottomrule
\end{tabular}

\vspace{0.5cm}


\begin{tabular}{P{4cm}P{2cm}P{8cm}}
	\multicolumn{3}{c}{\textbf {\large {LineaProduttiva} $\gets$ (\emph{ProduzioneVino, Imbottigliamento, Magazzino})}} \\
	\toprule
	\rowcolor[rgb]{.929, .929, .929} Attributo & Tipo & Descrizione \\
	\midrule
	\textbf{Id} & INTEGER &  Rappresenta univocamente la linea produttiva\\
	\bottomrule
\end{tabular}

\vspace{0.5cm}

\begin{tabular}{P{4cm}P{2cm}P{8cm}}
	\multicolumn{3}{c}{\textbf {\large {Magazzino} $\gets$ (\emph{MagBianco, MagRosso, MagRosato, MagSpumante})}} \\
	\toprule
	\rowcolor[rgb]{.929, .929, .929} Attributo & Tipo & Descrizione \\
	\midrule
	\textbf{NumBottiglie} & INTEGER &  Rappresenta la quantità delle bottiglie di vino, divise per tipologia di colore, contenute nel magazzino\\
	\bottomrule
\end{tabular}

\vspace{0.5cm}

\begin{tabular}{P{4cm}P{2cm}P{8cm}}
	\multicolumn{3}{c}{\textbf {\large {Dipendente}}} \\
	\toprule
	\rowcolor[rgb]{.929, .929, .929} Attributo & Tipo & Descrizione \\
	\midrule
	\textbf{CodiceFiscale} & VARCHAR &  Rappresenta univocamente il dipendente\\
	\midrule
	Nome & VARCHAR & Rappresenta il nome del dipendente \\
	\midrule
	Cognome & VARCHAR & Rappresenta il congnome del dipendente \\
	\bottomrule
\end{tabular}

\vspace{0.5cm}

\begin{tabular}{P{4cm}P{2cm}P{8cm}}
	\multicolumn{3}{c}{\textbf {\large {Macchinario}}} \\
	\toprule
	\rowcolor[rgb]{.929, .929, .929} Attributo & Tipo & Descrizione \\
	\midrule
	\textbf{Id} & INTEGER &  Rappresenta univocamente il macchinario\\
	\midrule
	Nome & VARCHAR & Rappresenta il nome commerciale del macchinario \\
	\midrule
	DataProssimaManutenzione & DATE & Rappresenta la data prossima della manutenzione \\
	\bottomrule
\end{tabular}

\end{center}
    
\subsection{Analisi delle relazioni e delle cardinalità}
    \begin{itemize}
	\item \underline{Vigneto - TipoUva}: \textbf{Proviene}
	
	\begin{itemize}
		\item Un vigneto produce solamente una tipologia d'uva $(0,1)$.
		\item Un tipo di uva può provenire da più vigneti $(1,N)$.
	\end{itemize}
	
\end{itemize}

\begin{itemize}
	\item \underline{Uva - Vino}: \textbf{Prodotto}
	
	\begin{itemize}
		\item Da un tipo di uva (di una determinata annata) possono essere stati prodotti più tipi di vino o nessun tipo di vino $(0,N)$*.
		\item Un vino è prodotto da un solo tipo d'uva (di una determinata annata) $(0,1)$.
	\end{itemize}
	
\end{itemize}

\begin{verse}
	*\emph{(0,N) perchè una nuova tipologia d'uva può essere appena stata acquistata e quindi non è stato prodotto ancora nessun vino.}
\end{verse}


\begin{itemize}
	\item \underline{MateriaPrima - Fornitore}: \textbf{Fornitura}*
	
	\begin{itemize}
		\item Una Materia prima viene fornita da un solo fornitore $(0,1)$.
		\item Un fornitore fornisce una o più tipologie di materie prime $(1,N)$.
	\end{itemize}
	
\end{itemize}

\begin{verse}
*\emph{Nella relazione sono presenti gli attributi \textbf{DataAcquisto, Prezzo, Quantità} perchè una fornitura può essere effettuata più volte, con data, prezzo e quantità differenti.}
\end{verse}


\begin{itemize}
	\item \underline{BottigliaDiVino - Vino}: \textbf{TipoVino}
	
	\begin{itemize}
		\item Una bottiglia di vino contiene solamente una tipologia di vino $(0,1)$.
		\item Un vino può essere contenuto in più bottiglie di vino o in nessuna $(0,N)$*.
	\end{itemize}
	
\end{itemize}

\begin{verse}
	*\emph{$(0,N)$ perchè una tipologia di vino può essere appena stato prodotto e quindi non è stato ancora imbottigliato, oppure possono essere terminate le bottiglie che lo contenevano.}
\end{verse}

\begin{itemize}
	\item \underline{BottigliaDiVino - Tappo}: \textbf{TipoTappo}
	
	\begin{itemize}
		\item Una bottiglia di vino ha solamente un tappo $(0,1)$.
		\item Un tipo di tappo può appartenere a più bottiglie di vino o a nessuna $(0,N)$*.
	\end{itemize}
	
\end{itemize}

\begin{verse}
	*\emph{(0,N) perchè una tipologia di tappo può essere appena stata acquistata e quindi non ancora assegnata a nessuna bottiglia di vino.}
\end{verse}

\begin{itemize}
	\item \underline{BottigliaDiVino - Bottiglia}: \textbf{TipoBottiglia}
	
	\begin{itemize}
		\item Una bottiglia di vino è formata da una sola bottiglia $(0,1)$.
		\item Un tipo di bottiglia può appartenere a più bottiglie di vino o a nessuna $(0,N)$*.
	\end{itemize}
	
\end{itemize}

\begin{verse}
	*\emph{(0,N) perchè una tipologia di bottiglia può essere appena stata acquistata e quindi non ancora assegnata a nessuna bottiglia di vino.}
\end{verse}

\begin{itemize}
	\item \underline{BottigliaDiVino - Magazzino}: \textbf{Conservata}
	
	\begin{itemize}
		\item Una tipologia di bottiglia di vino è conservata in un unico magazzino $(0,1)$
		\item Un magazzino contiene da una a molteplici tipologie di vino, poiche' la loro divisione avviene per colore e non per annata $(1,N)$.
	\end{itemize}
	
\end{itemize}

\begin{itemize}
	\item \underline{BottigliaDiVino - Ordine}: \textbf{Dettaglio}*
	
	\begin{itemize}
		\item Un tipo di bottiglia di vino può appartenere a zero o più ordini di vendita $(0,N)$.
		\item Un ordine di vendita contiene uno o più tipi di bottiglie di vino acquistate $(1,N)$.
	\end{itemize}
	
\end{itemize}

\begin{verse}
	*\emph{Nella relazione è presente l'attributo \textbf{QuantitàBottiglie} poiche' serve sapere il numero di bottiglie di vino per tipologia richieste dall'ordine.}
\end{verse}

\begin{itemize}
	\item \underline{Ordine - Corriere}: \textbf{Spedizione}*
	
	\begin{itemize}
		\item Un ordine viene spedito da un singolo corriere $(0,1)$.
		\item Un corriere può spedire uno o piu' ordini $(1,N)$.
	\end{itemize}
	
\end{itemize}

\begin{verse}
	*\emph{Nella relazione sono presenti gli attributi \textbf{DataSpedizione, Prezzo, DataArrivo} perchè dell'ordine è importante tracciare la data in cui è stato spedito l'ordine, il prezzo della spedizione e la data di consegna dell'ordine.}
\end{verse}

\begin{itemize}
	\item \underline{Ordine - Acquirente}: \textbf{Venduta}
	
	\begin{itemize}
		\item Un ordine e' effettuato da un singolo acquirente $(0,1)$.
		\item Un acquirente può effettuare uno o più ordini $(1,N)$.
	\end{itemize}
	
\end{itemize}

\begin{itemize}
	\item \underline{NegozioInterno - Evento}: \textbf{Ospita}*
	
	\begin{itemize}
		\item Un negozio interno può ospitare o nessuno o molteplici eventi $(0,N)$.
		\item Un evento può essere ospitato in uno o in molteplici negozi $(1,N)$.
	\end{itemize}
	
\end{itemize}

\begin{verse}
	*\emph{Nella relazione è presente l'attributo \textbf{Data} perchè un evento può essere ospitato in un negozio in una sola determinata data.}
\end{verse}


\begin{itemize}
	\item \underline{Evento - Vino}: \textbf{TemaVino}
	
	\begin{itemize}
		\item Un evento può avere uno o molteplici vini come tema $(1,N)$.
		\item Un vino può essere essere il tema di nessuno o molteplici eventi $(0,N)$.
	\end{itemize}
	
\end{itemize}

\begin{itemize}
	\item \underline{LineaProduttiva - Dipendente}: \textbf{Turno}*
	
	\begin{itemize}
		\item In una linea produttiva possono lavorare uno o molteplici dipendenti $(1,N)$.
		\item Un dipendete può lavorare in una sola linea produttiva $(0,1)$.
	\end{itemize}
	
\end{itemize}

\begin{verse}
	*\emph{Nella relazione sono presenti gli attributi \textbf{InizioTurno, FineTurno} perchè è utile rappresentare il giorno e l'orario in cui un dipendente lavora. Entrambi gli attributi sono rappresentati attraverso il tipo DATETIME.}
\end{verse}

\begin{itemize}
	\item \underline{LineaProduttiva - Dipendente}: \textbf{Diretto}
	
	\begin{itemize}
		\item Una linea produttiva è diretta da un solo dipendente $(0,1)$.
		\item Un dipendete può dirigere più linee produttive o nessuna $(0,N)$.
	\end{itemize}
	
\end{itemize}

\begin{itemize}
	\item \underline{Dipendente - Dipendente}: \textbf{Referisce}
	
	\begin{itemize}
		\item Un dipendente può essere riferito da uno o nessun altro dipendente $(0,1)$.
		\item Un dipendente può riferire molteplici dipendenti o nessuno $(0,N)$.
	\end{itemize}
	
\end{itemize}

\begin{itemize}
	\item \underline{LineaProduttiva - Macchinario}: \textbf{Utilizzo}
	
	\begin{itemize}
		\item In una linea produttiva possono essere utilizzati molteplici macchinari oppure nessuno $(0,N)$.
		\item Un macchinario può appartenere solamente ad una linea produttiva $(0,1)$.
	\end{itemize}
	
\end{itemize}

\begin{itemize}
	\item \underline{Macchinario - AziendaManutenzione}: \textbf{Manutenzione}*
	
	\begin{itemize}
		\item Un macchinario può aver ricevuto molteplici manutenzioni o nessuna $(0,N)$.
		\item Un'azienda di manutenzione può revisionare uno o molteplici macchinari $(1,N)$.
	\end{itemize}
	
\end{itemize}

\begin{verse}
	*\emph{Nella relazione sono presenti gli attributi \textbf{Costo} e \textbf{Data} che identificano il costo della singola manutenzione e in quale data e' stata effettuata.}
\end{verse}



\subsection{Diagramma ER}
    \begin{figure}
    \includegraphics[width=25cm,keepaspectratio,angle=90]{src/progettazioneConcettuale/assests/cantina_ER.png}
\end{figure}


\section{Progettazione logica}
\section{Analisi ridondanze}
	Lo schema concettuale rappresenta 2 attributi ridondanti, il primo è il \textbf{PrezzoTotale} nell'entità \emph{Ordine}. Infatti quest'attributo è derivabile da \textbf{Prezzo $\times$ QuantitàBottiglie $+$ Prezzo}, dove il primo Prezzo è l'attributo dell'entià \emph{Bottiglia di vino}, il secondo Prezzo è l'attributo della relazione \emph{Spedizione} e QuantitàBottiglie è l'attributo della relazione \emph{Dettaglio}. Da notare che quest'operazione matematica può essere ripetuta molteplici volte in base a quante tipologie diverse di bottiglie vengono acquistate nello stesso ordine. L'operazione coinvolta è la \emph{Stampa resoconto guadagno}, che avviene 1 volta alla settimana. Analizziamo la tavola dei volumi e delle operazioni:

\begin{center}
	\begin{tabular}{P{2cm}P{8cm}P{4cm}}
		\multicolumn{3}{c}{\textbf {\large {Tavola dei volumi}}} \\
		\toprule
		\rowcolor[rgb]{.929, .929, .929} Concetto & Costrutto & Volume \\
		\midrule
		BottigliaDiVino & E & 50\\
		\midrule
		Ordini & E & 350\\
		\midrule
		Corriere & E & 25\\
		\midrule
		Dettaglio & R & 350\\
		\midrule
		Spedizione & R & 350\\
		\bottomrule
	\end{tabular}

	\vspace{0.5cm}

	\textbf{\large{Tavola delle operazioni}}\\
	\vspace{0.2cm}
	\begin{minipage}{6cm}
		\rightline{
			\begin{tabular}{P{2cm}P{2cm}P{1cm}P{1cm}}
				\multicolumn{4}{c}{\textbf {Con ridondanza}} \\
				\toprule
				\rowcolor[rgb]{.929, .929, .929} Concetto & Costrutto & Accesso & Tipo \\
				\midrule
				BottigliaDiVino & E & 0 & -\\
				\midrule
				Ordini & E & 1 & L\\
				\midrule
				Dettaglio & R & 0 & -\\
				\midrule
				Spedizione & R & 0 & -\\
				\bottomrule
			\end{tabular}
		}
	\end{minipage}
	\hspace{2mm}
	\begin{minipage}{6cm}
		\leftline{
			\begin{tabular}{P{2cm}P{2cm}P{1cm}P{1cm}}
				\multicolumn{4}{c}{\textbf {Senza ridondanza}} \\
				\toprule
				\rowcolor[rgb]{.929, .929, .929} Concetto & Costrutto & Accesso & Tipo \\
				\midrule
				BottigliaDiVino & E & 350 & L\\
				\midrule
				Ordini & E & 1 & L\\
				\midrule
				Dettaglio & R & 350 & L\\
				\midrule
				Spedizione & R & 350 & L\\
				\bottomrule
			\end{tabular}
		}
	\end{minipage}

\end{center}

\begin{flushleft}
In quest'esempio vediamo che nello schema \emph{con ridondanza} in \emph{Ordine} avviene 1 accesso e 350 letture a settimana alla tabella \emph{Ordine} per stampare il resoconto settimanale (semplice somma di tutti i prezzi totali degli ordini di una specifica settimana).\\
\vspace{0.5cm}
Nello schema \emph{senza ridondanza}, invece, per ogni ordine bisognerà accedere alle relativa tabelle:
\begin{itemize}
	\item \emph{BottigliaDiVino} per conoscere il prezzo della singola bottiglia dell'ordine;
	\item \emph{Spedizione} per conoscere il prezzo della spedizione;
	\item \emph{Dettaglio} per conoscere il prezzo totale delle bottiglie acquistate.
\end{itemize}
Si avranno (350 $\times$ 3) $\times$ 350 dove (350 $\times$ 3) sono gli accessi totali alle tre tabelle citate prime e 350 sono le letture della tabella \emph{Ordini}. In totale si avranno 367500 letture.\\
\vspace{0.5cm}
In conclusione, dopo quest'analisi, si è deciso di mantenere l'attributo \textbf{Prezzo totale} in \emph{Ordine}.
\end{flushleft}

	
\section{Eliminazioni generalizzazioni}
Lo schema concenttuale presenta molte generalizzazioni. Si procede all'analisi di queste ultime per permettere la traduzione verso lo schema logico.\\


\begin{flushleft}
	\textbf{\large{MateriaPrima}}\\
	L'entità \emph{MateriaPrima} è una generalizzazione completa delle entità \emph{Uva, Bottiglia, Tappo}. Queste ultime presentano un carico di attributi e di relazioni maggiore rispetto a \emph{MateriaPrima}. Si è deciso di eliminare la generalizzazione e di creare una relazione specifica per \emph{Uva, Bottiglia, Tappo} con l'entità \emph{Fornitore}.
\end{flushleft}

\begin{flushleft}
	\textbf{\large{Fornitore}}\\
	L'entità \emph{Fornitore} non ha attributi ed è generalizzata dall'entità \emph{Azienda}. A sua volta \emph{Fornitore} è una generalizzazione di \emph{FornitoreUva, FornitoreTappi, FornitoreBottiglie}. Si è deciso di mantenere l'entità \emph{Fornitore} (avendo relazioni specifiche) e di eliminare \emph{FornitoreUva, FornitoreTappi, FornitoreBottiglie}. E' quindi stato aggiunto l'attributo \textbf{Tipologia} per identificare il tipo di fornitura di ogni fornitore.
\end{flushleft}


\begin{flushleft}
	\textbf{\large{Acquirente}}\\
	L'entità \emph{Acquirente} presenta un attributo ed una relazione. A loro volta anche le entità figlie \emph{Azienda e Privato} hanno attributi significativi. Per ristrutturare questa generalizzazione si è deciso di creare due relazioni differenti. La prima si chiama \textbf{IsAzienda} dove la cardinalità da \emph{Azienda} ad \emph{Acquirente} è $(0,1)$ poiche' un'azienda puo' non essere un acquirente, mentre da \emph{Acquirente} ad \emph{Azienda} è $(0,1)$ poiche' non tutti gli acquirenti sono un'azienda.
	La seconda relazione si chiama \textbf{IsPrivato} ed e' stata creata utilizzando lo stesso ragionamento di IsAzienda, ovvero la cardinalita' da \emph{Acquirente} a \emph{Privato} e' $(0,1)$ poiche' non tutti gli acquirenti sono privati, mentre da \emph{Privato} ad \emph{Acquirente} è $(1,1)$ poiche' un privato e' sempre un acquirente. Le entità \emph{Azienda} e \emph{Privato} mantengono gli attributi visti in precedenza, in piu' ad \emph{Acquirente} verra' aggiunto l'attributo \emph{NumAcquirente} che sara' chiave referenziale ad \emph{Acquirente}. Si e' deciso di aggiungere questo attributo consapevoli dell'eventualita' che quest'ultimo possa essere \textbf{NULL} visto che il numero di aziende che non sono acquirenti (es. fornitori) e' minore di quelle che lo sono.
\end{flushleft}

\begin{flushleft}
	\textbf{\large{Azienda}}\\
	L'entità \emph{Azienda} puo' identificare un fornitore, un'azienda di manutenzione o un negozio interno. Per ristrutturare questa generalizzazione si è deciso di aggiungere una relazione chiamata \emph{IsNegozioInterno} con cardinalità $(0,1)$ da \emph{Azienda} a \emph{NegozioInterno} poiche' non tutte le aziende sono un \emph{NegozioInterno}. La cardinalità da \emph{NegozioInterno} ad \emph{Azienda} è $(1,1)$ poiche' un negozio interno è anch'esso un'azienda. E' stata scelta questa soluzione perchè il numero di aziende che non sono negozi interni è maggiore rispetto al numero dei negozi interni, perciò è stata esclusa l'aggiunta di un attributo in \emph{Azienda} che identifichi se quest'ultima e' anche negozio interno. La stessa soluzione e' stata adottata anche per la generalizzazione con \emph{Fornitore}. Nella generalizzazione di \emph{AzManutenzione}, invece, si è deciso di togliere \emph{AzManutenzione} e di non specificare se un'azienda è un' azienda che effettua la manutenzione dei macchinari, ma di ricavarlo tramite la relazione \emph{Manutenzione}.
\end{flushleft}

\begin{flushleft}
	\textbf{\large{LineaProduttiva}}\\
	L'entità \emph{LineaProduttiva} è una generalizzazione a più livelli, infatti essa generalizza le entità \emph{ProduzioneVino, Imbottigliamento, Magazzino}. A sua volta l'entità \textbf{ProduzioneVino} generalizza \emph{IngrMateriePrime, Pigiatura, Fermentazione, Vinificazione, Svinatura}. Infine l'entità \emph{Magazzino} generalizza le entità \emph{MagBianco, MagSpumante, MagRosato, MagRosso}. Essendo tutte generalizzazioni complete e non avendo attributi (a parte l'entità \emph{Magazzino}*) che vanno a particolarizzare le varie entità figlie, si è deciso di aggiungere un attributo \textbf{Nome} nell'entità \emph{LineaProduttiva} che identifica queste specializzazioni.
\end{flushleft}

\begin{verse}
	*L'attributo \textbf{NumBottiglie} dell'entità \emph{Magazzino}, essendo di cardinalità $(1,1)$ per l'entità \emph{BottigliaDiVino}, è stato assegnato a quest'ultima rinominandolo \textbf{NumBottiglieMagazzino} perchè viene identificato univocamento per ogni tipologia di bottiglia di vino.
\end{verse}

\section{Partizionamento/accorpamento di entità e relationship}
\textbf{\large{Informazione}}\\
Si è deciso di aggiungere l'entità \emph{Informazione} in modo da raggruppare le (numerose) informazioni comuni delle entità \emph{Privato} ed \emph{Azienda} . L'entità \emph{Informazione} contiene quindi gli attributi \textbf{ID (VARCHAR), Email (VARCHAR), Telefono (VARCHAR), Nome (VARCHAR)}, i quali vengono rimossi dalle sopracitate entità.

\begin{flushleft}
\textbf{\large{Indirizzo}}\\
Si è deciso di aggiungere l'entità \emph{Indirizzo} in modo da raggruppare le informazioni dell'attributo \textbf{Indirizzo} di \emph{Informazione} e \emph{Vigneto}. L'entità \emph{Indirizzo} contiene quindi gli attributi \textbf{ID (VARCHAR), Stato (VARCHAR), Città (VARCHAR), Provincia (VARCHAR), CAP (INTEGER), Via (VARCHAR), NumCivico (INTEGER)}, i quali vengono rimossi dalle sopracitate entità.
\end{flushleft}

\begin{flushleft}
\textbf{\large{Date, InizioTurno, FineTurno}}\\
\textbf{InizioTurno, FineTurno} e più generalmente qualsiasi entità \textbf{\emph{Data}} è stata trasformata in un attributo singolo e rappresentata attraverso il tipo \textbf{DATE} o \textbf{DATETIME}.
\end{flushleft}


\section{Scelta degli identificatori principali}
Oltre agli identificatori visti nella sezione~\ref{analisi_entita}, sono stati aggiunti nuovi identificatori primari:

\begin{itemize}
	\item Si è deciso di aggiungere l'attributo \textbf{ID} (che funge da chiave primaria) e l'attributo \textbf{Nome} alle entità figlie \emph{FornitoreUva, FornitoreTappi, FornitoreBottiglie};
	\item Alle entità \emph{Azienda e Priavto} è stato aggiunto l'attributo \textbf{ID} che ha il ruolo di chiave primaria e di chiave referenziale con l'attributo \textbf{ID} di \emph{Acquirente};
	\item L'entità \emph{NegozioInterno} avrà un attributo \textbf{ID} che sarà chiave primaria e chiave referenziale con l'attributo \textbf{ID} di \emph{Azienda};
	\item L'entità \emph{Indirzzo} appena aggiunta avrà come chiave primaria l'attributo \textbf{ID} che sarà anche chiave referenziale o con l'attributo \textbf{ID} di \emph{Acquirente o Vigneto};
	\item E' stato aggiunto l'attributo \textbf{ID} a \emph{Uva, Tappo, Bottiglia, Evento} per rendere più semplice le relazioni con queste entità.
\end{itemize}

\section{Schema logico-relazionale e vincoli d'integrità referenziale}
\begin{center}
	\begin{minipage}[t]{7.5cm}
		\rightline{
			\begin{tabular}{P{7.5cm}}
				\toprule
				\rowcolor[rgb]{.929, .929, .929} \textbf{BottiglieDiVino} (\underline{Id}, Vino, Annata, Prezzo, NumBottiglieVendute, NumBottiglieMagazzino, NumBottiglieProdotte, IdTappo, IdBottiglia, IdMagazzino) \\
				\midrule
				(BottiglieDiVino.Vino, BottiglieDiVino.Annata) $\to$ (Vini.Nome, Vini.Uva.Annata)*                                                                                                                                     \\
				\midrule
				BottiglieDiVino.IdTappo $\to$ Tappi.Id                                                                                                                                                                                 \\
				\midrule
				BottiglieDiVino.IdBottiglia $\to$ Bottiglie.Id                                                                                                                                                                         \\
				\midrule
				BottiglieDiVino.IdMagazzino $\to$ LineeProduttive.Id                                                                                                                                                                   \\                                  
				\midrule
				\rowcolor[rgb]{.929, .929, .929} \textbf{Vini} (\underline{Nome}, GradazioneAlcolica, TempoFermentazione, StatoProduzione, Uva, Classificazione)                                                                                        \\
				\midrule
				Vini.Uva $\to$ Uva.Id                                                                                                                                                                                                  \\
				\midrule
				\rowcolor[rgb]{.929, .929, .929} \textbf{Uva} (\underline{Id}, TipoUva, Fornitore, Annata)                                                                                                                             \\
				\midrule
				Uva.Fornitore $\to$ Fornitori.Id                                                                                                                                                                                       \\
				\midrule
				Uva.TipoUva $\to$ TipiUva.Nome                                                                                                                                                                                         \\
				\midrule
				
			\end{tabular}
		}
	\end{minipage}
	\hspace{5mm}
	\begin{minipage}[t]{7.5cm}
		\rightline{
			\begin{tabular}{P{7.5cm}}
				\toprule
				\rowcolor[rgb]{.929, .929, .929} \textbf{Vigneti} (\underline{Id}, Indirizzo, TipoUva)                      \\
				\midrule
				Vigneti.Indirizzo $\to$ Indirizzi.Id                                                                        \\
				\midrule
				Vigneti.TipoUva $\to$ TipiUva.Nome                                                                          \\                                
				\midrule
				\rowcolor[rgb]{.929, .929, .929} \textbf{Tappi} (\underline{Id}, Forma, Materiale, Quantita, Fornitore)     \\
				\midrule
				Tappi.Fornitore $\to$ Fornitori.Id                                                                          \\                                
				\midrule
				\rowcolor[rgb]{.929, .929, .929} \textbf{Bottiglie} (\underline{Id}, Capacita, Colore, Quantita, Fornitore) \\
				\midrule
				Bottiglie.Fornitore $\to$ Fornitori.Id                                                                      \\                                
				\midrule
				\rowcolor[rgb]{.929, .929, .929} \textbf{FornituraUva} (\underline{Uva, DataAcquisto}, Prezzo, Quantita)    \\
				\midrule
				FornituraUva.Uva $\to$ Uva.Id                                                                               \\                                
				\midrule
				\rowcolor[rgb]{.929, .929, .929} \textbf{TipiUva} (\underline{Nome}, Colore)                                                                           \\                                
				\midrule
				\rowcolor[rgb]{.929, .929, .929} \textbf{Corrieri} (\underline{Id}, Nome)                                                                              \\                                
				\midrule
				\rowcolor[rgb]{.929, .929, .929} \textbf{Eventi} (\underline{Id}, Titolo, Edizione)                                                                    \\
				\midrule
				\rowcolor[rgb]{.929, .929, .929} \textbf{Acquirenti} (\underline{Id})                                                 \\        \midrule                        
			\end{tabular}
		}
	\end{minipage}
\end{center}


\begin{center}
	\begin{minipage}[t]{7.5cm}
		\rightline{
			\begin{tabular}{P{7.5cm}}
				\toprule
				\rowcolor[rgb]{.929, .929, .929} \textbf{FornituraTappi} (\underline{Tappo, DataAcquisto}, Prezzo, Quantita)          \\
				\midrule
				FornituraTappi.Tappo $\to$ Tappi.Id                                                                                   \\                               
				\midrule
				\rowcolor[rgb]{.929, .929, .929} \textbf{FornituraBottiglie} (\underline{Bottiglia, DataAcquisto}, Prezzo, Quantita)                                   \\
				\midrule
				FornituraBottiglie.Bottiglia $\to$ Bottiglie.Id                                                                                                        \\                                
				\midrule
				\rowcolor[rgb]{.929, .929, .929} \textbf{Dipendenti} (\underline{CodiceFiscale}, Nome, Cognome, Referente)            \\
				\midrule
				Dipendenti.Referente $\to$ Dipendenti.CodiceFiscale                                                                   \\                                
				\midrule
				\rowcolor[rgb]{.929, .929, .929} \textbf{Turni} (\underline{Dipendente, InizioTurno}, FineTurno, LineaProduttiva)                                      \\
				\midrule
				Turni.Dipendente $\to$ Dipendenti.CodiceFiscale                                                                                                        \\
				\midrule
				Turni.LineaProduttiva $\to$ LineeProduttive.Id                                                                                                         \\                                
				\midrule
				\rowcolor[rgb]{.929, .929, .929} \textbf{Fornitori} (\underline{Id}, Tipologia)                                       \\
				\midrule
				Fornitori.Id $\to$ Aziende.PartitaIVA \\                                
				\midrule
				\rowcolor[rgb]{.929, .929, .929} \textbf{NegoziInterni} (\underline{Id})                                              \\
				\midrule
				NegoziInterni.Id $\to$ Aziende.PartitaIVA \\                                
				\midrule
				\rowcolor[rgb]{.929, .929, .929} \textbf{Ospita} (\underline{Evento, Negozio}, Data)                                  \\
				\midrule
				Ospita.Evento $\to$ Eventi.Id                                                                                         \\
				\midrule
				Ospita.Negozio $\to$ NegoziInterni.Id                                                                                 \\                                
				\midrule
				\rowcolor[rgb]{.929, .929, .929} \textbf{TemiVino} (\underline{Vino, Evento})                                         \\
				\midrule
				TemiVino.Vino $\to$ Vini.Nome                                                                                         \\
				\midrule
				TemiVino.Evento $\to$ Eventi.Id                                                                                       \\                                
				\midrule
				\rowcolor[rgb]{.929, .929, .929} \textbf{Informazioni} (\underline{Id}, Nome, Telefono, Email, Indirizzo)                                              \\
				\midrule
				Informazioni.Indirizzo $\to$ Indirizzi.Id                                                                                                              \\        
				\midrule
				\rowcolor[rgb]{.929, .929, .929} \textbf{Ordini} (\underline{Id}, PrezzoTotale *, Data, Acquirente)                                                    \\
				\midrule
				Ordini.Acquirente $\to$ Acquirenti.Id                                                                                                                  \\                                
				\midrule
				
			\end{tabular}
		}
	\end{minipage}
	\hspace{5mm}
	\begin{minipage}[t]{7.5cm}
		\rightline{
			\begin{tabular}{P{7.5cm}}
				\toprule
				\rowcolor[rgb]{.929, .929, .929} \textbf{Manutenzioni} (\underline{Id}, Macchinario, Azienda, Costo, Data)            \\
				\midrule
				Manutenzioni.Macchinario $\to$ Macchinari.Id                                                                          \\
				\midrule
				Manutenzioni.Azienda $\to$ Aziende.PartitaIVA                                                                                 \\                                
				\midrule
				\rowcolor[rgb]{.929, .929, .929} \textbf{Macchinari} (\underline{Id}, Nome, DataProssimaManutenzione, DataAcquisto, LineaProduttiva)                   \\
				\midrule
				Macchinari.LineaProduttiva $\to$ LineeProduttive.Id                                                                                                    \\                                
				\midrule
				\rowcolor[rgb]{.929, .929, .929} \textbf{Dettagli} (\underline{Ordine, BottigliaDiVino}, QuantitaBottiglie)                      \\
				\midrule
				Dettagli.Ordine $\to$ Ordini.Id                                                                                       \\
				\midrule
				Dettagli.BottigliaDiVino $\to$ BottiglieDiVino.Id                                                                                \\                                
				\midrule
				\rowcolor[rgb]{.929, .929, .929} \textbf{LineeProduttive} (\underline{Id}, Nome, Direttore)                           \\
				\midrule
				LineeProduttive.Direttore $\to$ Dipendenti.CodiceFiscale                                                              \\                                
				\midrule
				\rowcolor[rgb]{.929, .929, .929} \textbf{Spedizioni} (\underline{Ordine, Corriere}, DataSpedizione, DataConsegna, Prezzo)                              \\
				\midrule
				Spedizioni.Ordine $\to$ Ordini.Id                                                                                                                      \\
				\midrule
				Spedizioni.Corriere $\to$ Corrieri.Id                                                                                                                  \\                                
				\midrule
				\rowcolor[rgb]{.929, .929, .929} \textbf{Privati} (\underline{Id}, Cognome, InformazioniAggiuntive)                                                    \\
				\midrule
				Privati.Id $\to$ Acquirenti.Id                                                                                                                         \\
				\midrule
				Privati.InformazioniAggiuntive $\to$ Informazioni.Id                                                                                                   \\                                                        
				\midrule
				\rowcolor[rgb]{.929, .929, .929} \textbf{Aziende} (\underline{PartitaIVA}, NumAcquirente, NomeReferente, CognomeReferente, InformazioniAggiuntive) \\
				\midrule
				Aziende.NumAcquirente $\to$ Acquirenti.Id                                                                                                              \\
				\midrule
				Aziende.InformazioniAggiuntive $\to$ Informazioni.Id                                                                                                   \\                                
				\midrule
				\rowcolor[rgb]{.929, .929, .929} \textbf{Indirizzi} (\underline{Id}, Via, NumeroCivico, Stato, Provincia, Citta, CAP) \\
				
				\midrule
			\end{tabular}
		}
	\end{minipage}
	
	
	\begin{verse}
	*\emph{Per ovviare al problema del vincolo referenziale tra 	\textbf{BottiglieDiVino.Annata e Vini.Uva.Annata}, oltre che di \textbf{Ordini.PrezzoTotale} come somma di $BottiglieDiVino.Prezzo \times Dettagli.QuantitaBottiglie)+ Spedizioni.Prezzo$} sarebbe necessario aggiungere dei \textbf{TRIGGER}. Questi non sono stati implementati poiche' l'argomento non e' stato trattato durante il corso.
\end{verse}

\end{center}


\section{Query ed indici}
\subsection{Query}
\begin{enumerate}
	\item \textbf{Stampa resoconto guadagno.}\\
	      \begin{tabularx}{\textwidth}{|X|X|}
			  \hline
			  \vspace{.01mm}
		      SELECT
		      SUM(Ordini.PrezzoTotale)
		      FROM
		      Ordini
		      WHERE
		      Ordini.Data <= CURRENT\_TIMESTAMP
		      AND Ordini.Data >= DATE\_ADD(CURRENT\_TIMESTAMP, INTERVAL -7 DAY);
			   &
			   \raisebox{-\totalheight}{\includegraphics[width=0.47\textwidth, keepaspectratio]{src/queryIndici/assets/Query1.png}}
		      \\
		      \hline
	      \end{tabularx}
	\item \textbf{Bottiglia/e di vino più venduta/e dell'anno 2019 con la relativa quantità.}\\
	      \begin{tabularx}{\textwidth}{|X|X|}
		      \hline
			  \vspace{.01mm}
		      CREATE VIEW OrdineQuantita AS
		      SELECT
		      Vini.Nome,
		      SUM(Dettagli.QuantitaBottiglie) AS BottiglieVendute
		      FROM Dettagli,
		      BottiglieDiVino,
		      Vini, Ordini
		      WHERE
		      Dettagli.BottigliaDiVino = BottiglieDiVino.Id
		      AND Vini.Nome = BottiglieDiVino.Vino AND DATE\_FORMAT(Ordini.Data, '\%Y') = '2019'
		      AND Ordini.Id = Dettagli.Ordine
		      GROUP BY
		      Dettagli.BottigliaDiVino;
		      \newline\newline
		      SELECT
		      OrdineQuantita.Nome,
		      OrdineQuantita.BottiglieVendute
		      FROM
		      OrdineQuantita
		      WHERE
		      OrdineQuantita.BottiglieVendute IN (
		      SELECT
		      MAX(OrdineQuantita.BottiglieVendute)
		      FROM
		      OrdineQuantita
		      );
			   &
			   \raisebox{-\totalheight}{\includegraphics[width=0.47\textwidth , keepaspectratio]{src/queryIndici/assets/Query2.png}}
		      \\
		      \hline
		  \end{tabularx}
		  \vspace{1cm}
	\item \textbf{Lista dei vini prodotti con la relativa tipologia di uva utilizzata e il fornitore di quest'ultima.}\\
	      \begin{tabularx}{\textwidth}{|X|X|}
		      \hline
			  \vspace{.01mm}
		      SELECT DISTINCT
		      Vini.Nome AS Vino,
		      Uva.TipoUva as TipoUva,
		      Informazioni.Nome as Fornitore
		      FROM
		      Vini,
		      Uva,
		      Informazioni,
		      Aziende
		      WHERE
		      Vini.Uva = Uva.Id
		      AND Uva.Fornitore = Aziende.PartitaIVA
		      AND Aziende.InformazioniAggiuntive = Informazioni.Id;
			   &
			   \raisebox{-\totalheight}{\includegraphics[width=0.47\textwidth , keepaspectratio]{src/queryIndici/assets/Query3.png}}
			   \\
		      \hline
	      \end{tabularx}
	\item \textbf{Lista dei dipendenti (ordinati in ordine alfabetico) che sono supervisori di altri dipendenti.}\\
	      \begin{tabularx}{\textwidth}{|X|X|}
		      \hline
			  \vspace{.01mm}
		      SELECT
		      DISTINCT D1.Nome,
		      D1.Cognome
		      FROM
		      Dipendenti as D1,
		      Dipendenti as D2
		      WHERE
		      D2.Referente = D1.CodiceFiscale
		      ORDER BY
		      D1.Nome,
		      D1.Cognome;
			   &
			   \hspace{1.8cm}
			   \raisebox{-\totalheight}{\includegraphics[width=0.3\textwidth]{src/queryIndici/assets/Query4Limited.png}}
		      \\
		      \hline
	      \end{tabularx}
	\item \textbf{Lista dei dipendenti (nome, cognome) che hanno lavorato il giorno 22 ottobre 2019, con inizio e fine turno, ordinati in modo decrescente in base all'inizio del turno.}\\
	      \begin{tabularx}{\textwidth}{|X|X|}
		      \hline
			  \vspace{.01mm}
		      SELECT
		      D.Nome,
		      D.cognome,
		      T.InizioTurno,
		      T.FineTurno
		      FROM
		      Dipendenti as D,
		      Turni as T
		      WHERE
		      T.Dipendente = D.CodiceFiscale
		      AND DATE\_FORMAT(T.InizioTurno, '\%Y-\%m-\%d') = '2019-10-22'
		      ORDER BY
		      T.InizioTurno DESC;
			   &
			   \raisebox{-\totalheight}{\includegraphics[width=0.47\textwidth]{src/queryIndici/assets/Query5.png}}
		      \\
		      \hline
	      \end{tabularx}
	\item \textbf{Lista degli acquirenti che hanno acquistato il maggior valore di bottiglie di vino dalla cantina.}\\
	      \begin{tabularx}{\textwidth}{|X|X|}
		      \hline
			  \vspace{.01mm}
		      CREATE VIEW SpeseTotali AS
		      SELECT
		      Ordini.Acquirente,
		      SUM(Ordini.PrezzoTotale) AS SpesaTotale
		      FROM
		      Ordini
		      GROUP BY
		      Ordini.Acquirente;
		      \newline\newline
		      SELECT
		      SpeseTotali.Acquirente,
		      SpeseTotali.SpesaTotale
		      FROM
		      SpeseTotali
		      WHERE
		      SpeseTotali.SpesaTotale IN (
		      SELECT
		      MAX(SpeseTotali.SpesaTotale)
		      FROM
		      SpeseTotali
		      );
			   &
			   \raisebox{-\totalheight}{\includegraphics[width=0.47\textwidth]{src/queryIndici/assets/Query6.png}}
		      \\
		      \hline
	      \end{tabularx}
\end{enumerate}

\subsection{Indici}
<<<<<<< HEAD
Analizzando il traffico di informazioni da e verso la base di dati ci si e' resi conto che la tabella \emph{Dipendenti} contiene dati che sono acceduti di frequente in lettura e molto di rado (solamente 10 volte l'anno) in scrittura. Per aumentare l'efficienza della base di dati si e' quindi deciso di creare un Indice nella suddetta tabella.
=======
È opportuno creare un indice sulla tabella \emph{Dipendenti} perchè analizzando il carico della base di dati, tale tabella contiene dati che vengono acceduti spesso in lettura ma non in scrittura. L'unico momento in cui la tabella viene acceduta in scrittura è in concomitanza all'assunzione di nuovi dipendenti, azione svolta solamente 10 volte l'anno.
>>>>>>> 3c80bc4750c85057c30313b686631f981adf0aa6
\begin{flushleft}
	\textbf{{Creazione indice:}} \emph{CREATE INDEX idx\_dipendenti ON Dipendenti (CodiceFiscale)}
\end{flushleft}


\end{document}