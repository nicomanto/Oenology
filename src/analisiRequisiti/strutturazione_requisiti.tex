\begin{center}
	\begin{tabular}{P{16cm}}
		\toprule
		\rowcolor[rgb]{.929, .929, .929} \textbf {\large {FRASI RELATIVE A BOTTIGLIA DI VINO}} \\
		\midrule
		Una bottiglia di vino e' il prodotto finale che viene venduto dalla cantina, e rappresenta l'unione di una bottiglia di vetro, di un tappo e di un tipo di vino. Ogni bottiglia e' caratterizzata dal suo prezzo, dal numero di unita' uguali prodotte e vendute, oltre che dal tipo di tappo e dal tipo di bottiglia 
		Altri dati importanti sono il prezzo per bottiglia, il numero di bottiglie prodotte e vendute, il tipo di tappo e bottiglia ed infine dalla classificazione: vini a denominazione d'origine controllata e garantita (D.O.C.G.); vini a denominazione d'origine controllata (D.O.C.); vini ad indicazione geografica tipica (I.G.T.). Ciascuna tipologia di bottiglia di vino è racchiusa in magazzini. Ovviamente le bottiglie di vino possono essere spedite in una determinata quantità in un ordine di vendita\\
		\bottomrule
	\end{tabular}

	\vspace{0.5cm}
	
	\begin{tabular}{P{16cm}}
		\toprule
		\rowcolor[rgb]{.929, .929, .929} \textbf {\large {FRASI RELATIVE A TIPO DI VINO}} \\
		\midrule
		Relativamente ai tipi di vino rappresentiamo il nome, la gradazione alcolica, il tempo di fermentazione e l'uva utilizzata oltre che lo stato produttivo, ovvero se il vino viene prodotto ancora o e' ormai fuori commercio. \\
		\bottomrule
	\end{tabular}

	\vspace{0.5cm}
	
	\begin{tabular}{P{16cm}}
		\toprule
		\rowcolor[rgb]{.929, .929, .929} \textbf {\large {FRASI RELATIVE A MATERIA PRIMA}} \\
		\midrule
		Le materie prima che vengono acquistate viene rappresentato un ID univoco che le rappresenta. Se la materia prima e' l'Uva questa viene rappresentata dall'anno di raccolta, dal nome e dal colore, se, invece, la materia prima sono i Tappi questa viene rappresentata dalla forma e dal materiale, infine se la materia prima sono le Bottiglie, questa viene rappresentata dalla capacita' e dal colore.\\
		\bottomrule
	\end{tabular}

	\vspace{0.5cm}
	
	\begin{tabular}{P{16cm}}
		\toprule
		\rowcolor[rgb]{.929, .929, .929} \textbf {\large {FRASI RELATIVE A VIGNETO}} \\
		\midrule
		I Vigneti sono rappresentati da un ID univoco e dalla loro posizione geografica\\
		\bottomrule
	\end{tabular}
	
	\vspace{0.5cm}
	
	\begin{tabular}{P{16cm}}
		\toprule
		\rowcolor[rgb]{.929, .929, .929} \textbf {\large {FRASI RELATIVE AD AZIENDA}} \\
		\midrule

		Una Azienda e' rappresentata dal nome, dalla Partita IVA, dal nome e dal cognome del referente oltre che dal numero di telefono, dalla email aziendale e dalla sua posizione geografica. Per le aziende che sono Fornitori rappresentiamo poi la tipologia di Materia Prima fornita, sia essa Uva, Tappi o Bottiglie.\\
		\bottomrule
	\end{tabular}
	
	\vspace{0.5cm}
	
	\begin{tabular}{P{16cm}}
		\toprule
		\rowcolor[rgb]{.929, .929, .929} \textbf {\large {FRASI RELATIVE AD ORDINE}} \\
		\midrule
		Relativamente ad un ordine rappresentiamo un ID univoco, la data in cui e' stato effettuato, il corriere a cui e' stata affidato, la data di spedizione, la data di consegna e il prezzo totale (che deve corrispondere al prodotto tra il prezzo delle bottiglie di vino acquistate e il relativo prezzo sommato al costo della spedizione).\\
		\bottomrule
	\end{tabular}
	
	\vspace{0.5cm}
	
	\begin{tabular}{P{16cm}}
		\toprule
		\rowcolor[rgb]{.929, .929, .929} \textbf {\large {FRASI RELATIVE AD PRIVATO}} \\
		Per i privati, che sono delle persone che hanno effettuato un ordine nel negozio online della cantina rappresentiamo il nome, il cognome, il numero di telefono e l'email oltre che l'indirizzo di residenza (che corrisponde con l'indirizzo di spedizione).\\
		\bottomrule
	\end{tabular}
	
	\vspace{0.5cm}
	
	\begin{tabular}{P{16cm}}
		\toprule
		\rowcolor[rgb]{.929, .929, .929} \textbf {\large {FRASI RELATIVE AD EVENTO}} \\
		Gli eventi hanno una tematica che è rappresentata da un nostro vino. Gli eventi vengono definiti tramite il titolo dell'evento e il numero di edizione. A quest'evento partecipano delle persone. Ovviamente ogni evento ha una data prefissata in cui si svolgerà.\\
		\bottomrule
	\end{tabular}
	
	\vspace{0.5cm}
	
	\begin{tabular}{P{16cm}}
		\toprule
		\rowcolor[rgb]{.929, .929, .929} \textbf {\large {FRASI RELATIVE A PARTECIPANTE}} \\
		A quest'evento partecipano delle persone che si devono registrare e inserire i propri dati (nome, cognome, età).\\
		\bottomrule
	\end{tabular}
	
	\vspace{0.5cm}
	
	\begin{tabular}{P{16cm}}
		\toprule
		\rowcolor[rgb]{.929, .929, .929} \textbf {\large {FRASI RELATIVE A LINEA PRODUTTIVA}} \\
		Riguardo al reparto produttivo sono presenti tutte le tipologie di lavorazione dell'uva per la produzione del vino. Queste linee produttive sono l'ingresso delle materie prime, la pigiatura, la fermentazione, la vinificazione e la svinatura. Oltre a queste esistono altri reparti produttivi come l'imbottigliamento e il magazzino citato all'inizio. Tutti questi reparti sono identificati tramite un identificativo. Ciascuna tipologia di bottiglia di vino è racchiusa in magazzini aventi il numero di bottiglie contenute divisi per la colorazione del vino (rosso, bianco, rosato e spumante). In ognugno di questi reparti lavorano dei dipendeti in un determinati turni di lavoro. Per ogni linea produttiva sono presenti dei macchinari che aiutano i dipendenti nelle mansioni quotidiane.\\
		\bottomrule
	\end{tabular}
	
	\vspace{0.5cm}
	
	\begin{tabular}{P{16cm}}
		\toprule
		\rowcolor[rgb]{.929, .929, .929} \textbf {\large {FRASI RELATIVE A DIPENDENTE}} \\
		Per ogni dipendete (identificato tramite il codice fiscale) viene registrato il nome e il cognome. Ogni dipendente avrà un supervisore a cui fare riferimento e ogni linea produttiva sarà diretta da un dipendente dell'azienda.\\
		\bottomrule
	\end{tabular}
	
	\vspace{0.5cm}
	
	\begin{tabular}{P{16cm}}
		\toprule
		\rowcolor[rgb]{.929, .929, .929} \textbf {\large {FRASI RELATIVE A MACCHINARIO}} \\
		I macchinari sono identificati tramite un codice univoco. Ad ogni macchinario viene fatta una manutenzione periodica, quindi è utile tenere traccia della data della prossima manutenzione. I macchinari avranno anche un nome commerciale e la data di acquisto. Le manutenzioni avranno un costo e verrano eseguite in una determinata data. Le manutenzioni vengono eseguite da aziende esterne.\\
		\bottomrule
	\end{tabular}
	
\end{center}